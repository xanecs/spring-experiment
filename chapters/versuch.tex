\chapter{Versuch}

Beim Versuch wurden Federn mit verschiedenen Federhärten mit verschiedenen Massen und verschiedenen Amplituden zum Schwingen gebracht. Mit einem Messwerterfassungssystem (CASSY) wurde die Auslenkung in Abhängigkeit von der Zeit aufgezeichnet.

Von den verschiedenen Federn wurde jeweils durch Messen der Auslenkung, die durch Massestücke verschiedener Masse hervorgerufen wird, die Federhärte bestimmt.

Anhand der Messswerte wurde
\begin{itemize}
    \item der Einfluss der Amplitude auf die Periodendauer
    \item der Zusammenhang $T \sim \sqrt{m}$
    \item der Zusammenhang $T \sim \frac{1}{D}$
\end{itemize}
untersucht.

Zusätzlich wurde anhand der aufgezeichneten Daten die schwingende Masse bestimmt, und mit der Masse des Massestücks verglichen.

Zu den Ergebnissen wurde jeweils eine Fehlerbetrachtung angestellt.

\section{Material}
\begin{itemize}
    \item Verschiedene Federn
    \item Massestücke bekannter Masse
    \item Stativmaterial
    \item CASSY mit Laptop und BMW-Box
    \item Faden und leichtes Gegengewicht
\end{itemize}

\section{Aufbau}
Zunächst wurde mit dem Stativmaterial eine geeignete Befestigung für die Federn aufgebaut. Die Feder hängt oben an einem Haken, unten wird sie nicht befestigt, so dass sie frei schwingen kann. Neben der Feder wird der Bewegungs-Messwandler (BMW) angebracht. Das leichte Gegengewicht wurde an den Faden gebunden, dieser wurde über den BMW gelegt und sein anderes Ende an die Unterseite der Feder gebunden. Ebenfalls an die untere Seite der Feder wurde das Massestück angehängt.